\documentclass[paper=a4, fontsize=11pt]{scrartcl} 
\usepackage[utf8]{inputenc}
\usepackage{amsmath}
\usepackage{amsfonts}
\usepackage{amssymb}
\author{Kim Thuong Ngo}


\usepackage[T1]{fontenc} 
\usepackage{fourier} 

\usepackage{lipsum} 

\usepackage{listings}
\usepackage{graphicx}
\usepackage{tabularx}

\usepackage{sectsty}
\allsectionsfont{\centering \normalfont\scshape} 

\usepackage{fancyhdr} 
\pagestyle{fancyplain} 
\fancyhead{}
\fancyfoot[L]{} 
\fancyfoot[C]{} 
\fancyfoot[R]{\thepage} 
\renewcommand{\headrulewidth}{0pt} 
\renewcommand{\footrulewidth}{0pt}
\setlength{\headheight}{13.6pt}

\numberwithin{equation}{section} 
\numberwithin{figure}{section} 
\numberwithin{table}{section}

\setlength\parindent{0pt} 

\newcommand{\horrule}[1]{\rule{\linewidth}{#1}} 

\title{	
\normalfont \normalsize 
\textsc{Grundlagen des Internets} \\ [25pt] 
\horrule{0.5pt} \\[0.4cm] 
\huge Aufgaben \\ 
\horrule{2pt} \\[0.5cm] 
}

\author{Kim Thuong Ngo} 

\date{\normalsize\today} 

%----------------------------------------------------------------------------------------

\begin{document}

\maketitle 

\newpage

\tableofcontents

\newpage

%----------------------------------------------------------------------------------------

\section{ISO/OSI-Schichtenmodell}
\subsection{1.}
Ordnen Sie folgende Protkolle, Spezifikationen und Medien den Schicvhten 1-4 oder 7 des ISO/OSI-Models zu. \\
Schicht 1: Koaxialkabel, Sichtverbindung\newline
Schicht 2:\newline
Schicht 3: IP\newline
Schicht 4: TCP, UDP\newline
Schicht 7: SMTP, HTTP, VoIP\newline

\subsection{2.}
Auf welchen der oben genannten Schichten "operieren" folgende Geräte und Standards? \\
Schicht 1: Hub\newline
Schicht 2: Bridge, PPP, IEEE 802.3, IEEE 802.11\newline
Schicht 3: Router, Mobile Ip\newline
Schicht 4: \newline
Schicht 7: Web Browser\newline

\subsection{3.}
Was versteht man unter horizontaler und was unter vertikaler Kommunikation im ISO/OSI-Modell? \\
Man versteht unter horizontaler Kommunikation im OSI/ISO-Modell, dass Protokolle nur in einer einzelnen Schicht ausgetauscht werden. Es werden keine Protokolle mit anderen Schichten ausgetauscht. Dafür interagieren (vertikale Kommunikation) die Schichten mit Diensten untereinander. Um diese Kommunikationsarten zu ermöglichen werden versandten Nachrichten bei jeder Schicht Header (teils auch Trailer) angehängt.\newline
\includegraphics[width=15cm,height=10cm]{Bild.jpeg}

\section{Einkapselung von Daten} 
Welcher Anteil der Netzbandbreite wird von den Headern belegt? \\
Die belegte Netzbandbreite setzt sich aus der Nachricht M und den n Headern(h) zusammen (M + n * h)\newline
$\dfrac{hn}{M+hn}$ Byte (in Prozent ausgedrückt) ist der Anteil der Netzbandbreite, welche von den Headern belegt wird.

\section{Protokollschichten}
\subsection{1.}
Geben Sie an, welche Protokolle hierbei an den Punkten A-D auf den Schichten 2-4 im ISO/OSI-Modell verwendet werden! \\

\begin{tabular}{c|c|c|c|c}
& A & B & C & D \\ \hline
Protokollschicht 4 & TCP & & & TCP \\ \hline
Protokollschicht 3 & IP & & IP & IP \\ \hline
Protokollschicht 2 &  & Ethernet & Ethernet & Ethernet \\ \hline
\end{tabular}

\subsection{2.}
Geben Sie an, welche Quell- und Zieladressen auf Schicht 2 und 3 und welchen Zielport auf Schicht 4 die Protkollheader jeweils an den Punkten A-D enthalten! \\

\begin{tabular}{c|c|c|c|c}
& A & B & C & D \\ \hline
Zielport (Schicht 4) &  &  &  &  \\ \hline
Quelladresse Schicht 3 & 193.196.30.23 & 193.196.30.23 & 193.196.30.23 & 193.196.30.23 \\ \hline
Zieladresse Schicht 3 & 134.2.5.1 & 134.2.5.1 & 134.2.5.1 & 134.2.5.1 \\ \hline
Quelladresse Schicht 2 & 42:59:3b:54:8a:91 & 42:59:3b:54:8a:91 & 42:59:3b:54:8a:91 & 42:3c:be:a8:56:fa \\ \hline
Zieladresse Schicht 2 & 42:fd:12:5e:3d:09 & 42:fd:12:5e:3d:09 & 42:fd:12:5e:3d:09 & \\ \hline
\end{tabular}

\subsection{3.}
Geben sie die jeweils höchste Protokollschicht an, die auf den jeweiligen Geräten verarbeitet wird! \\

\begin{itemize}
\item Laptop $\rightarrow$ Layer 7
\item WLAN-Access-Point $\rightarrow$ Layer 2
\item Ethernet Switch $\rightarrow$ Layer 2
\item IP-Router $\rightarrow$ Layer 3
\item Web-Server $\rightarrow$ Layer 7
\end{itemize}

%----------------------------------------------------------------------------------------

\section{Virtual Circuits}
Geben Sie die Label-Forwardingtabellen der Router R1-R6 mit den Spalten In-Port,In-Label, Out-Port, Out Label an. \\

R1:\newline
\begin{tabular}{c|c|c|c}
In-Port & In-Label & Out-Port & Out-Label \\\hline
- & - & IF0 & L1 \\
\end{tabular}\newline
R2:\newline
\begin{tabular}{c|c|c|c}
In-Port & In-Label & Out-Port & Out-Label \\\hline
- & - & IF0 & L4 \\
\end{tabular}\newline
R3:\newline
\begin{tabular}{c|c|c|c}
In-Port & In-Label & Out-Port & Out-Label \\\hline
- & - & IF0 & L2\\
\end{tabular}\newline
R4:\newline
\begin{tabular}{c|c|c|c}
In-Port & In-Label & Out-Port & Out-Label \\\hline
IF0 & L5 & IF1 & L6\\\hline
IF0 & L5 & IF2 & L2\\\hline
IF1 & L1 & IF0 & L3\\\hline
IF2 & L4 & IF0 & L3\\
\end{tabular}\newline
R5:\newline
\begin{tabular}{c|c|c|c}
In-Port & In-Label & Out-Port & Out-Label \\\hline
IF0 & L6 & IF1 & L5\\\hline
IF0 & L6 & IF2 & L4\\\hline
IF1 & L3 & IF0 & L1\\\hline
IF2 & L2 & IF0 & L1\\
\end{tabular}\newline
R6:\newline
\begin{tabular}{c|c|c|c}
In-Port & In-Label & Out-Port & Out-Label \\\hline
- & - & IF0 & L6\\
\end{tabular}\newline

%--------------------------------------------
\section{Label Stacking}
\subsection{1.}
Forwarding ohne LSPs \\

R1:\newline
\begin{tabular}{c|c|c|c}
In-Port & In-Label & Out-Port & Out-Label \\\hline
- & - & FI0 & L0\\
\end{tabular}\newline
(A) Label Stack: L0, L1, L2, L3\newline
R2:\newline
\begin{tabular}{c|c|c|c}
In-Port & In-Label & Out-Port & Out-Label \\\hline
IF0 & L0 & IF1 & L1\\
\end{tabular}\newline
(B)Label Stack: L1, L2, L3\newline
R3:\newline
\begin{tabular}{c|c|c|c}
In-Port & In-Label & Out-Port & Out-Label \\\hline
IF0 & L1 & IF1 & L2\\
\end{tabular}\newline
(C)Label Stack: L2, L3\newline
R4:\newline
\begin{tabular}{c|c|c|c}
In-Port & In-Label & Out-Port & Out-Label \\\hline
IF1 & L2 & IF0 & L3\\
\end{tabular}\newline
(D)Label Stack: L3\newline
R5:\newline
\begin{tabular}{c|c|c|c}
In-Port & In-Label & Out-Port & Out-Label \\\hline
IF0 & L3 & - & -\\
\end{tabular}\newline

\subsection{2.}
Forwarding über einen LSP von R1 bis R5 \\

R1:\newline
\begin{tabular}{c|c|c|c}
In-Port & In-Label & Out-Port & Out-Label \\\hline
- & - & IF0 & L0\\
\end{tabular}\newline
(A) Label Stack: L0\newline
R2:\newline
\begin{tabular}{c|c|c|c}
In-Port & In-Label & Out-Port & Out-Label \\\hline
IF0 & L0 & IF1 & L1\\
\end{tabular}\newline
(B)Label Stack: L1\newline
R3:\newline
\begin{tabular}{c|c|c|c}
In-Port & In-Label & Out-Port & Out-Label \\\hline
IF0 & L1 & IF1 & L2\\
\end{tabular}\newline
(C)Label Stack: L2\newline
R4:\newline
\begin{tabular}{c|c|c|c}
In-Port & In-Label & Out-Port & Out-Label \\\hline
IF0 & L2 & IF1 & L3\\
\end{tabular}\newline
(D)Label Stack: L3\newline
R5:\newline
\begin{tabular}{c|c|c|c}
In-Port & In-Label & Out-Port & Out-Label \\\hline
IF0 & L3 & - & -\\
\end{tabular}\newline

\subsection{3.}
Forwarding über einen LSP von R2 bis R4

R1:\newline
\begin{tabular}{c|c|c|c}
In-Port & In-Label & Out-Port & Out-Label \\\hline
- & - & FI0 & L0\\
\end{tabular}\newline
(A) Label Stack: L0, L3\newline
R2:\newline
\begin{tabular}{c|c|c|c}
In-Port & In-Label & Out-Port & Out-Label \\\hline
IF0 & L0 & IF1 & L1\\
\end{tabular}\newline
(B)Label Stack: L1, L3\newline
R3:\newline
\begin{tabular}{c|c|c|c}
In-Port & In-Label & Out-Port & Out-Label \\\hline
IF0 & L1 & IF1 & L2\\
\end{tabular}\newline
(C)Label Stack: L2, L3\newline
R4:\newline
\begin{tabular}{c|c|c|c}
In-Port & In-Label & Out-Port & Out-Label \\\hline
IF1 & L2 & IF0 & L3\\
\end{tabular}\newline
(D)Label Stack: L3\newline
R5:\newline
\begin{tabular}{c|c|c|c}
In-Port & In-Label & Out-Port & Out-Label \\\hline
IF0 & L3 & - & -\\
\end{tabular}\newline

\subsection{4.}
Forwarding über einen LSP von R1 bis R5, der seinerseits zwischen R2 bis R4 über einen weiteren LSP läuft \\

R1:\newline
\begin{tabular}{c|c|c|c}
In-Port & In-Label & Out-Port & Out-Label \\\hline
- & - & FI0 & L0\\
\end{tabular}\newline
(A) Label Stack: L0\newline
R2:\newline
\begin{tabular}{c|c|c|c}
In-Port & In-Label & Out-Port & Out-Label \\\hline
IF0 & L0 & IF1 & L1\\
\end{tabular}\newline
(B)Label Stack: L1\newline
R3:\newline
\begin{tabular}{c|c|c|c}
In-Port & In-Label & Out-Port & Out-Label \\\hline
IF0 & L1 & IF1 & L2\\
\end{tabular}\newline
(C)Label Stack: L2\newline
R4:\newline
\begin{tabular}{c|c|c|c}
In-Port & In-Label & Out-Port & Out-Label \\\hline
IF1 & L2 & IF0 & L3\\
\end{tabular}\newline
(D)Label Stack: L3\newline
R5:\newline
\begin{tabular}{c|c|c|c}
In-Port & In-Label & Out-Port & Out-Label \\\hline
IF0 & L3 & - & -\\
\end{tabular}\newline

%--------------------------------------------
\section{IPv4-Adressen}

\subsection{1.}
Wie lang darf das IP-Präfix höchstens sein um die gewünschte Unterteilung in Subnetze zu ermöglichen? Geben Sie Netzwerkadresse und das Präfix an und begründen Sie ihre Antwort. \\

Präfix: 10 \\
Netzwerkadresse: 192.168.1.0 \\

Das Netzwerk muss /23 höchstens sein, weil da 512 Hosts reinpassen. 

\subsection{2.}
Weisen Sie jeder Rechner Klasse einen IP-Adressbereich zu! \\

\begin{itemize}
\item Linux Server \\
D $\rightarrow$ 192.168.3.52-102/24 \\
\item Linux PCs \\
B $\rightarrow$ 192.168.3.1-51 /24 \\
\item Windows PCs \\
A $\rightarrow$ 192.168.1.1-201 /24 \\ 
C $\rightarrow$ 192.168.2.1-201/24 \\ 
\end{itemize}

%----------------------------------------------------------------------------------------
\section{Forwardingtabellen}
Geben Sie die Forwarding Information Base (FIB) zu den Hosts A-F und den Routern R1,R2 und R3 an. Versuchen Sie, soweit wie möglich, Routen zusammenzufassen. Verwenden Sie IF1 als Bezeichnung für das Interface, das mit dem Backbone-Netz verbunden ist und IF0 für die Interfaces im jeweiligen LAN. \\

\begin{tabular}{|c|}
\hline
A\\
LAN1:192.168.0.1\\
Routes\\
192.168.0.2 via LAN1\\
192.168.0.254 via LAN1\\
134.2.0.0/16 via 192.168.0.254\\
LAN2 via 192.168.0.254\\
LAN3 via 192.168.0.254
\\\hline
\end{tabular}\newline\newline

\begin{tabular}{|c|}
\hline
B\\
LAN1:192.168.0.1\\
192.168.0.1 via LAN1\\
192.168.0.254 via LAN1\\
134.2.0.0/16 via 192.168.0.254\\
10.0.8.0/21 via 192.168.0.254\\
10.0.0.0/8 via 192.168.0.254
\\\hline
\end{tabular}\newline\newline

\begin{tabular}{|c|}
\hline
C\\
LAN2:10.0.8.2\\
Routes\\
10.0.8.3 via LAN2\\
10.0.8.1 via LAN2\\
134.2.0.0/16 via 10.0.8.1\\
192.168.0.0/24 via 10.0.8.1\\
10.0.0.0/8 via 10.0.8.1
\\\hline
\end{tabular}\newline\newline

\begin{tabular}{|c|}
\hline
D\\
LAN2:10.0.8.3\\
Routes\\
10.0.8.2 via LAN2\\
10.0.8.1 via LAN2\\
134.2.0.0/16 via 10.0.8.1\\
192.168.0.0/24 via 10.0.8.1\\
10.0.0.0/8 via 10.0.8.1
\\\hline
\end{tabular}\newline\newline

\begin{tabular}{|c|}
\hline
E\\
LAN3:10.0.0.1\\
Routes\\
10.0.0.2 via LAN3\\
10.0.0.254 via LAN3\\
134.2.0.0/16 via 10.0.0.254\\
192.168.0.0/24 via 10.0.0.254\\
10.0.8.0/21 via 10.0.0.254
\\\hline
\end{tabular}\newline\newline

\begin{tabular}{|c|}
\hline
F\\
LAN3:10.0.0.1\\
Routes\\
10.0.0.1 via LAN3\\
10.0.0.254 via LAN3\\
134.2.0.0/16 via 10.0.0.254\\
192.168.0.0/24 via 10.0.0.254\\
10.0.8.0/21 via 10.0.0.254
\\\hline
\end{tabular}\newline\newline

R1\newline
\begin{tabular}{|c|c|c|}
\hline
Prefix & Gateway & Interface\\\hline
192.168.0.0/24 & - & IF0\\
134.2.0.0/16 & - & IF1\\
0/0 & 134.2.0.2 & IF1\\
0/0 & 134.2.0.3 & IF1\\
10.0.8.0/21 & 134.2.0.3 & IF1\\
10.0.0.0/8 & 134.2.0.2 & IF1
\\\hline
\end{tabular}\newline\newline

R2\newline
\begin{tabular}{|c|c|c|}
\hline
Prefix & Gateway & Interface\\\hline
10.0.0.0/8 & - & IF0\\
134.2.0.0/16 & - & IF1\\
0/0 & 134.2.0.1 & IF1\\
0/0 & 134.2.0.3 & IF1\\
192.168.0.0/24 & 134.2.0.1 & IF1\\
10.0.8.0/21 & 134.2.0.3 & IF1
\\\hline
\end{tabular}\newline\newline

R3\newline
\begin{tabular}{|c|c|c|}
\hline
Prefix & Gateway & Interface\\\hline
10.0.8.0/21 & - & IF0\\
134.2.0.0/16 & - & IF1\\
0/0 & 134.2.0.1 & IF1\\
0/0 & 134.2.0.2 & IF1\\
192.168.0.0/24 & 134.2.0.1 & IF1\\
10.0.0.0/8 & 134.2.0.2 & IF1
\\\hline
\end{tabular}\newline\newline

%--------------------------------------------
\section{IP-Fragmentierung}
Zwei Hosts A und B kommunizieren über IP miteinander. Jeder Host befindet sich in seinem eigenen Netzwerk und ist mit dem Netzwerk des anderen Hosts über ein drittes Netzwerk verbunden. Die Netzwerke sind untereinander mit Routern verbunden. Jedes Netzwerk verwendet seine eigene Übertragstechnik, wodurch sich unterschiedliche Maximum Transmission Units (MTUs) ergeben.

\subsection{1.}
Finden in beide Richtungen Fragmentierungen statt? Begründen Sie Ihre Antwort! \\

Nein, es finden nur Fragmentierungen von Host A zu Host B statt. \\

Begründung:
Host B ist an einem Netzwerk mit einem MTU-Wert von 1500 angeschlossen. Sendet also Host B an Host A ein Paket, dann werden nur maximal 1500 Byte große Fragmente versendet, die Host A ohne Probleme aufnehmen kann. In die andere Richtung ist dies nicht möglich, d.h. der IP-Router muss die Fragmente in weitere kleinere Fragmente aufteilen und sendet die Fragmente getrennt. D.h. Router A muss das Paket fragmentieren, um es zu Router B zu senden und Router B muss die Fragmente fragmentieren, ums sie zum Ziel Host B zu senden.

\subsection{2.}
Host A sendet ein 4352 Byte großes IP-Paket an Host B. Was geschieht auf dem Weg zu Host B mit dem Paket? Wie oft wird das Paket fragmentiert? Geben Sie die IP Fragmente, die bei Host B ankommen sowie die Zwischenschritte als Tabelle mit den Spalten fragment number,payload, fragment length, fragment offset und MF-bit an. \\

Host A sendet ein 4352 Byte großes IP-Paket an Host B. Netz A ist ein FDDI, d.h. das Paket bekommt erstmal einen FDDI Header. Jetzt wird das Paket mit dem Header an Router A gesendet. Dieser entpackt es und wirft den Rahmen Header weg. Er fragmentiert das Paket in zwei Fragmente mit gleichem Format, da der MTU-Wert nur 2272 beträgt und setzt einen WiFi-Header auf beide Fragmente. Ein Bit im Feld flags kennzeichnet, dass es zwei Fragmente sind und weitere Felder des Headers kennzeichnen auch die Reihenfolge,wie das Paket ursprünglich zusammengehört. Um ein Paket für die Übertragung in ein anderes Netzwerk zu fragmentieren, berechnet der Router die maximale Datenmenge pro Fragment und die benötigte Anzahl von Fragmenten anhand des MTU-Wertes. Beim Erzeugen der Fragmente, beginnt der Router jedes Fragment mit einer Kopie des Original Headers anzuheften und modifiziert die einzelnen Header-Felder, um jedes Fragment entsprechend zu kennzeichnen. Schließlich kopiert er die Daten aus dem Original-Paket und überträgt sie einzeln.
Router A sendet nun beide Fragmente zu Router B. Dieser muss die Fragmente nochmals fragmentieren, da der MTU-Wert wieder geringer ist als die Fragmente. Der WiFi Header wird wieder verworfen und ein Ethernet Header wird verwendet. Dabei fragmentiert er die Fragmente neu, wie oben beschrieben. Es sind jetzt also vier Fragmente die dann über das Netz B an den Host B gesendet werden. Nach der Übertragung müssen die einzelnen Fragmente wieder zusammengesetzt werden. Da jedes Fragment mit dem Header des Original-Paketes beginnen, haben alle Fragmente die selbe Zieladresse. Darüber hinaus befindet im Header, des letzten Fragments ein zusätzliches Bit, um zu erkennen ob alle Fragmente angekommen sind. Der Ziel-Host B sammelt die Fragmente und fügt sie zum Original zusammen. D.h. auf dem Weg wird das Paket zweimal fragmentiert und beim Ziel Host B reassembliert. \\

\begin{tabular}{|c|c|c|c|c|c|}
\hline
 & fragment number &  payload & fragment length & fragment offset & MF-bit \\\hline
FDDI & 1 & 4352 & 4352 & 0 & 0\\\hline
WiFi & 1.1 & 2176 & 2272 & 0 & 1\\
     & 1.2 & 2176 & 2272 & 272 & 0\\\hline
Ethernet & 1.1.1 & 1088 & 1500 & 0 & 3\\
         & 1.1.2 & 1088 & 1500 & 0 & 0\\
         & 1.2.1 & 1088 & 1500 & 0 & 0\\
         & 1.2.2 & 1088 & 1500 & 136 & 0\\\hline
\end{tabular}


%--------------------------------------------
\section{Address Resolution Protocol (ARP)}
Gegeben sei ein IP-Netzwerk mit den Host A,B,C und leeren ARP-Tabellen. A möchte ein Paket an B senden und verschickt zuvor einen ARP-Request. Anschließend möchte C ein Paket an A senden.

\subsection{1.}
Muss C einen eigenen ARP-Request verschicken? Begründen Sie Ihre Antwort! \\

Ja, C muss eine ARP-Request versenden. \\

Begründung:
A möchte ein Paket an B senden. A sendet also eine ARP-Request mit der IP-Adresse von B an alle im Netzwerk angeschlossenen Hosts. Alle Hosts erhalten die Anfrage, aber nur der Host B, mit der angefragten IP-Adresse, sendet ein ARP-Reply zurück. Alle Übrigen verwerfen die Anfrage, ohne eine Antwort zu senden. D.h. Host C muss eine seperate ARP-Request mit der IP-Adresse von A schicken, um das Paket senden zu können.

\subsection{2.}
Wie viele ARP-Pakete müssen mindestens verschickt werden, um den angestrebten Paketversand sowohl für A als auch für C zu ermöglichen? Nennen Sie auch die Paket-Typen (Request,Reply) und begründen Sie ihre Antwort! \\ 

\begin{tabular}{|c|c|c|}
\hline
A sendet ARP-Request &                        & \\\hline
                     & B erhält die Anfrage   & C erhält die Anfrage \\\hline
                     & B sendet ARP-Reply     & C verwirft die Anfrage \\\hline
A erhält die Antwort &                        & \\\hline
                     &                        & C sendet ARP-Request \\\hline
A erhält die Anfrage & B erhält die Anfrage   & \\\hline
A sendet ARP-Reply   & B verwirft die Anfrage & \\\hline
                     &                        & C erhält die Antwort  \\\hline
\end{tabular} \\


D.h. es müssen mindestens 6 ARP-Pakete verschickt werden, weil ARP-Requests an alle im Netzwerk gesendet werden müssen. Im Gegensatz dazu werden ARP-Replys nicht rundgesendet, sondern direkt vom angesprochenem Host an den anfragenden Host gesendet.

%--------------------------------------------

\end{document}