\documentclass[paper=a4, fontsize=11pt]{scrartcl} 
\usepackage[utf8]{inputenc}
\usepackage{amsmath}
\usepackage{amsfonts}
\usepackage{amssymb}
\author{Kim Thuong Ngo}


\usepackage[T1]{fontenc} 
\usepackage{fourier} 

\usepackage{lipsum} 

\usepackage{listings}
\usepackage{graphicx}
\usepackage{tabularx}

\usepackage{sectsty}
\allsectionsfont{\centering \normalfont\scshape} 

\usepackage{fancyhdr} 
\pagestyle{fancyplain} 
\fancyhead{}
\fancyfoot[L]{} 
\fancyfoot[C]{} 
\fancyfoot[R]{\thepage} 
\renewcommand{\headrulewidth}{0pt} 
\renewcommand{\footrulewidth}{0pt}
\setlength{\headheight}{13.6pt}

\numberwithin{equation}{section} 
\numberwithin{figure}{section} 
\numberwithin{table}{section}

\setlength\parindent{0pt} 

\newcommand{\horrule}[1]{\rule{\linewidth}{#1}} 

\title{	
\normalfont \normalsize 
\textsc{Grundlagen des Internets} \\ [25pt] 
\horrule{0.5pt} \\[0.4cm] 
\huge Aufgaben \\ 
\horrule{2pt} \\[0.5cm] 
}

\author{Kim Thuong Ngo} 

\date{\normalsize\today} 

%----------------------------------------------------------------------------------------

\begin{document}

\maketitle 

\newpage

\tableofcontents

\newpage

%----------------------------------------------------------------------------------------

\section{ISO/OSI-Schichtenmodell}
\subsection{1.}
Ordnen Sie folgende Protkolle, Spezifikationen und Medien den Schicvhten 1-4 oder 7 des ISO/OSI-Models zu. \\
Schicht 1: Koaxialkabel, Sichtverbindung\newline
Schicht 2:\newline
Schicht 3: IP\newline
Schicht 4: TCP, UDP\newline
Schicht 7: SMTP, HTTP, VoIP\newline

\subsection{2.}
Auf welchen der oben genannten Schichten "operieren" folgende Geräte und Standards? \\
Schicht 1: Hub\newline
Schicht 2: Bridge, PPP, IEEE 802.3, IEEE 802.11\newline
Schicht 3: Router, Mobile Ip\newline
Schicht 4: \newline
Schicht 7: Web Browser\newline

\subsection{3.}
Was versteht man unter horizontaler und was unter vertikaler Kommunikation im ISO/OSI-Modell? \\
Man versteht unter horizontaler Kommunikation im OSI/ISO-Modell, dass Protokolle nur in einer einzelnen Schicht ausgetauscht werden. Es werden keine Protokolle mit anderen Schichten ausgetauscht. Dafür interagieren (vertikale Kommunikation) die Schichten mit Diensten untereinander. Um diese Kommunikationsarten zu ermöglichen werden versandten Nachrichten bei jeder Schicht Header (teils auch Trailer) angehängt.\newline
\includegraphics[width=15cm,height=10cm]{Bild.jpeg}

\section{Einkapselung von Daten} 
Welcher Anteil der Netzbandbreite wird von den Headern belegt? \\
Die belegte Netzbandbreite setzt sich aus der Nachricht M und den n Headern(h) zusammen (M + n * h)\newline
$\frac{M+n*h}{100} * (n*h)$ Byte (in Prozent ausgedrückt) ist der Anteil der Netzbandbreite, welche von den Headern belegt wird.

\section{Protokollschichten}
\subsection{1.}
Geben Sie an, welche Protokolle hierbei an den Punkten A-D auf den Schichten 2-4 im ISO/OSI-Modell verwendet werden! \\

\begin{tabular}{c|c|c|c|c}
& A & B & C & D \\ \hline
Protokollschicht 4 & TCP & & & TCP \\ \hline
Protokollschicht 3 & IP & & IP & IP \\ \hline
Protokollschicht 2 & Ethernet & Ethernet & Ethernet & Ethernet \\ \hline
\end{tabular}

\subsection{2.}
Geben Sie an, welche Quell- und Zieladressen auf Schicht 2 und 3 und welchen Zielport auf Schicht 4 die Protkollheader jeweils an den Punkten A-D enthalten! \\

\begin{tabular}{c|c|c|c|c}
& A & B & C & D \\ \hline
Zielport (Schicht 4) &  &  &  &  \\ \hline
Quelladresse Schicht 3 & 193.196.30.23 & 193.196.30.23 & 193.196.30.23 & 193.196.30.23 \\ \hline
Zieladresse Schicht 3 & 134.2.5.1 & 134.2.5.1 & 134.2.5.1 & 134.2.5.1 \\ \hline
Quelladresse Schicht 2 & 42:59:3b:54:8a:91 & 42:59:3b:54:8a:91 & 42:59:3b:54:8a:91 & 42:3c:be:a8:56:fa \\ \hline
Zieladresse Schicht 2 & 42:fd:12:5e:3d:09 & 42:fd:12:5e:3d:09 & 42:fd:12:5e:3d:09 & 42:8e:63:d7:f8:b2 \\ \hline
\end{tabular}

\subsection{3.}
Geben sie die jeweils höchste Protokollschicht an, die auf den jeweiligen Geräten verarbeitet wird! \\

\begin{itemize}
\item Laptop $\rightarrow$ Layer 2
\item WLAN-Access-Point $\rightarrow$ Layer 3
\item Ethernet Switch $\rightarrow$ Layer 2
\item IP-Router $\rightarrow$ Layer 3
\item Web-Server $\rightarrow$ Layer 7
\end{itemize}

%----------------------------------------------------------------------------------------

\section{Virtual Circuits}
Geben Sie die Label-Forwardingtabellen der Router R1-R6 mit den Spalten In-Port,In-Label, Out-Port, Out Label an. \\

R1:\newline
\begin{tabular}{c|c|c|c}
In-Port & In-Label & Out-Port & Out-Label \\\hline
- & - & IF0 & L1 \\
\end{tabular}\newline
R2:\newline
\begin{tabular}{c|c|c|c}
In-Port & In-Label & Out-Port & Out-Label \\\hline
- & - & IF0 & L4 \\
\end{tabular}\newline
R3:\newline
\begin{tabular}{c|c|c|c}
In-Port & In-Label & Out-Port & Out-Label \\\hline
- & - & IF0 & L2\\
\end{tabular}\newline
R4:\newline
\begin{tabular}{c|c|c|c}
In-Port & In-Label & Out-Port & Out-Label \\\hline
IF0 & L5 & IF1 & L6\\\hline
IF0 & L5 & IF2 & L2\\\hline
IF1 & L1 & IF0 & L3\\\hline
IF2 & L4 & IF0 & L3\\
\end{tabular}\newline
R5:\newline
\begin{tabular}{c|c|c|c}
In-Port & In-Label & Out-Port & Out-Label \\\hline
IF0 & L6 & IF1 & L5\\\hline
IF0 & L6 & IF2 & L4\\\hline
IF1 & L3 & IF0 & L1\\\hline
IF2 & L2 & IF0 & L1\\
\end{tabular}\newline
R6:\newline
\begin{tabular}{c|c|c|c}
In-Port & In-Label & Out-Port & Out-Label \\\hline
- & - & IF0 & L6\\
\end{tabular}\newline

%--------------------------------------------
\section{Label Stacking}
\subsection{1.}
Forwarding ohne LSPs \\

R1:\newline
\begin{tabular}{c|c|c|c}
In-Port & In-Label & Out-Port & Out-Label \\\hline
- & - & FI0 & L0\\
\end{tabular}\newline
(A) Label Stack: L0, L1, L2, L3\newline
R2:\newline
\begin{tabular}{c|c|c|c}
In-Port & In-Label & Out-Port & Out-Label \\\hline
IF0 & L0 & IF1 & L1\\
\end{tabular}\newline
(B)Label Stack: L1, L2, L3\newline
R3:\newline
\begin{tabular}{c|c|c|c}
In-Port & In-Label & Out-Port & Out-Label \\\hline
IF0 & L1 & IF1 & L2\\
\end{tabular}\newline
(C)Label Stack: L2, L3\newline
R4:\newline
\begin{tabular}{c|c|c|c}
In-Port & In-Label & Out-Port & Out-Label \\\hline
IF1 & L2 & IF0 & L3\\
\end{tabular}\newline
(D)Label Stack: L3\newline
R5:\newline
\begin{tabular}{c|c|c|c}
In-Port & In-Label & Out-Port & Out-Label \\\hline
IF0 & L3 & - & -\\
\end{tabular}\newline

\subsection{2.}
Forwarding über einen LSP von R1 bis R5 \\

R1:\newline
\begin{tabular}{c|c|c|c}
In-Port & In-Label & Out-Port & Out-Label \\\hline
- & - & IF0 & L0\\
\end{tabular}\newline
(A) Label Stack: L0\newline
R2:\newline
\begin{tabular}{c|c|c|c}
In-Port & In-Label & Out-Port & Out-Label \\\hline
IF0 & L0 & IF1 & L1\\
\end{tabular}\newline
(B)Label Stack: L1\newline
R3:\newline
\begin{tabular}{c|c|c|c}
In-Port & In-Label & Out-Port & Out-Label \\\hline
IF0 & L1 & IF1 & L2\\
\end{tabular}\newline
(C)Label Stack: L2\newline
R4:\newline
\begin{tabular}{c|c|c|c}
In-Port & In-Label & Out-Port & Out-Label \\\hline
IF0 & L2 & IF1 & L3\\
\end{tabular}\newline
(D)Label Stack: L3\newline
R5:\newline
\begin{tabular}{c|c|c|c}
In-Port & In-Label & Out-Port & Out-Label \\\hline
IF0 & L3 & - & -\\
\end{tabular}\newline

\subsection{3.}
Forwarding über einen LSP von R2 bis R4

R1:\newline
\begin{tabular}{c|c|c|c}
In-Port & In-Label & Out-Port & Out-Label \\\hline
- & - & FI0 & L0\\
\end{tabular}\newline
(A) Label Stack: L0, L3\newline
R2:\newline
\begin{tabular}{c|c|c|c}
In-Port & In-Label & Out-Port & Out-Label \\\hline
IF0 & L0 & IF1 & L1\\
\end{tabular}\newline
(B)Label Stack: L1, L3\newline
R3:\newline
\begin{tabular}{c|c|c|c}
In-Port & In-Label & Out-Port & Out-Label \\\hline
IF0 & L1 & IF1 & L2\\
\end{tabular}\newline
(C)Label Stack: L2, L3\newline
R4:\newline
\begin{tabular}{c|c|c|c}
In-Port & In-Label & Out-Port & Out-Label \\\hline
IF1 & L2 & IF0 & L3\\
\end{tabular}\newline
(D)Label Stack: L3\newline
R5:\newline
\begin{tabular}{c|c|c|c}
In-Port & In-Label & Out-Port & Out-Label \\\hline
IF0 & L3 & - & -\\
\end{tabular}\newline

\subsection{4.}
Forwarding über einen LSP von R1 bis R5, der seinerseits zwischen R2 bis R4 über einen weiteren LSP läuft \\

R1:\newline
\begin{tabular}{c|c|c|c}
In-Port & In-Label & Out-Port & Out-Label \\\hline
- & - & FI0 & L0\\
\end{tabular}\newline
(A) Label Stack: L0\newline
R2:\newline
\begin{tabular}{c|c|c|c}
In-Port & In-Label & Out-Port & Out-Label \\\hline
IF0 & L0 & IF1 & L1\\
\end{tabular}\newline
(B)Label Stack: L1\newline
R3:\newline
\begin{tabular}{c|c|c|c}
In-Port & In-Label & Out-Port & Out-Label \\\hline
IF0 & L1 & IF1 & L2\\
\end{tabular}\newline
(C)Label Stack: L2\newline
R4:\newline
\begin{tabular}{c|c|c|c}
In-Port & In-Label & Out-Port & Out-Label \\\hline
IF1 & L2 & IF0 & L3\\
\end{tabular}\newline
(D)Label Stack: L3\newline
R5:\newline
\begin{tabular}{c|c|c|c}
In-Port & In-Label & Out-Port & Out-Label \\\hline
IF0 & L3 & - & -\\
\end{tabular}\newline

%--------------------------------------------
\section{IPv4-Adressen}

\subsection{1.}
Wie lang darf das IP-Präfix höchstens sein um die gewünschte Unterteilung in Subnetze zu ermöglichen? Geben Sie Netzwerkadresse und das Präfix an und begründen Sie ihre Antwort. \\

Präfix: /23 \\
Netzwerkadresse: 192.168.1.0 \\

Das Präfix muss /23 höchstens sein, weil da 512 Hosts reinpassen. 

\subsection{2.}
Weisen Sie jeder Rechner Klasse einen IP-Adressbereich zu! \\

\begin{itemize}
\item Linux Server \\
D $\rightarrow$ 192.168.3.52-102/26 \\
\item Linux PCs \\
B $\rightarrow$ 192.168.3.1-51 /26 \\
\item Windows PCs \\
A $\rightarrow$ 192.168.1.1-201 /24 \\ 
C $\rightarrow$ 192.168.2.1-201/24 \\ 
\end{itemize}

%----------------------------------------------------------------------------------------

\end{document}