\documentclass[paper=a4, fontsize=11pt]{scrartcl} 
\usepackage[utf8]{inputenc}
\usepackage{amsmath}
\usepackage{amsfonts}
\usepackage{amssymb}
\author{Kim Thuong Ngo}


\usepackage[T1]{fontenc} 
\usepackage{fourier} 

\usepackage{lipsum} 

\usepackage{listings}
\usepackage{graphicx}
\usepackage{tabularx}

\usepackage{sectsty}
\allsectionsfont{\centering \normalfont\scshape} 

\usepackage{fancyhdr} 
\pagestyle{fancyplain} 
\fancyhead{}
\fancyfoot[L]{} 
\fancyfoot[C]{} 
\fancyfoot[R]{\thepage} 
\renewcommand{\headrulewidth}{0pt} 
\renewcommand{\footrulewidth}{0pt}
\setlength{\headheight}{13.6pt}

\numberwithin{equation}{section} 
\numberwithin{figure}{section} 
\numberwithin{table}{section}

\setlength\parindent{0pt} 

\newcommand{\horrule}[1]{\rule{\linewidth}{#1}} 

\title{	
\normalfont \normalsize 
\textsc{Grundlagen des Internets} \\ [25pt] 
\horrule{0.5pt} \\[0.4cm] 
\huge Skript \\ 
\horrule{2pt} \\[0.5cm] 
}

\author{Kim Thuong Ngo} 

\date{\normalsize\today} 

%----------------------------------------------------------------------------------------

\begin{document}

\maketitle 

\newpage

\tableofcontents

\newpage

%----------------------------------------------------------------------------------------

\section{Grundlagen über Netzwerke}

%--------------------------------------------------------------------

\subsection{Kommunikationsnetzwerke und Internet}

Ein Netzwerk ist ein Set von Hardware und Software, dass Informationen weiterleitet.

Beispiele für Netzwerke:
\begin{itemize}
\item POTS (Plain old telephone system)
\item Broadcast networks (television, radio)
\item mobile telephone networks
\item computer networks (Intranet, Internet) 
\end{itemize}

$\Rightarrow$ Netzwerkevolution : Netzwerke vermischen sich

z.B.
\begin{itemize}
\item triple play (Internet access, speech, television delivered over IP)
\item Quadruple play (triple play over wireless)
\end{itemize}

%--------------------------------------------------------------------

\subsection{Klassifikation}

basierend auf den geographischen Umfang:
\begin{itemize}
\item sehr klein: Nano-Kommunikationsnetzwerk
\item $0,1 m$ - $1 m$: PAN (Personal Area Network), interner Bus/ Netzwerk
\item $10 m$ - $1 km$: LAN (Local Area Network)
\item $1 km$ - $100 km$: MAN (Metropolitan Area Network)
\item $100km$ und mehr: WAN (Wide Area Network)
\item noch mehr: Satelliten Netzwerk, interplanetare Netzwerke, ...
\end{itemize}

räumliche Struktur
\begin{itemize}
\item full-mesh 
belastbar auf Fehler, nicht ökonomisch für große Anzahl an Knoten, z.B. WiFi adhoc Netzwerk
\item Stern
zentral schaltender Knoten, ökonomisch, nicht belastbar, z.B. Gigabit Ethernet
\item Baum
hierarchisch mit Wurzeln, nicht belastbar, z.B. Telefonnetzwerk
\item Bus
benutzt als normales Übertragungsmedium, alles Stationen können Pakete senden und empfangen, keine Weiterleitung an Zwischengliedern benötigt, benötigt mehrere Zugangsprotokolle, z.B. Standard Ethernet
\item Ring
Knoten senden, empfangen und leiten Pakete weiter, Ränder zwischen Nachbarn, kann belastbar sein, wenn 2 Käbel in gegensätzliche Richtung agieren, benötigt mehrere Zugangsprotokolle, z.B. Token Ring
\end{itemize}

Übertragungsmodelle
\begin{itemize}
\item Unicast
Information sendet an einem speziellen Knoten (point-to-point), z.B. telephone
\item Broadcast
Information sendet an alle Knoten im Netzwerk, z.B. Radio
\item Multicast
Information sendet an alle Knoten in einer definierten Gruppe (point-to-multipoint), z.B. Audio-/Videokonferenzen
\item Anycast
Information sendet an einen Knoten in einer Gruppe von möglichen Adressaten
\end{itemize}

%--------------------------------------------------------------------

\subsection{Services, Layer und Protokolle}

%--------------------------------------------------------------------

\subsection{ISO / OSI Referenzmodelle}

%--------------------------------------------------------------------

\subsection{Internet Referenzmodell}

%--------------------------------------------------------------------

\subsection{Kritik am Referenzmodell}

%----------------------------------------------------------------------------------------
\newpage

\section{Netzwerk Layer}

%----------------------------------------------------------------------------------------
\newpage

\section{Transport Layer}

%----------------------------------------------------------------------------------------
\newpage

\section{Middleboxes}

%----------------------------------------------------------------------------------------
\newpage

\section{Anwendungs-Layer}

%----------------------------------------------------------------------------------------
\newpage

\section{Peer-to-Peer}

\end{document}